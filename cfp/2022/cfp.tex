\documentclass{../cfp}
\begin{document}

\PrintLogo{}

\PrintCity{15}{../kazan}

\PrintTitle{Second}

In cooperation
with
\href{https://www.sigplan.org}{ACM SIGPLAN},
\href{https://www.sigsoft.org}{ACM SIGSOFT},\newline
and
\href{https://conferences.ieee.org/conferences_events/conferences/conferencedetails/53703}{IEEE Computer Society}.

\vspace{6pt}

\PrintAddress
  {Innopolis, Kazan, Russia}
  {23-Apr-2022 (one day, \textbf{mostly online})}
  {https://easychair.org/cfp/ICCQ22}
  {EasyChair CFP}

\vspace{12pt}
\person{keynote/charles-zhang}{%
  \href{http://www.cs.ust.hk/~charlesz/}{Charles Zhang} \\
  \href{http://www.cs.ust.hk/}{HKUST} \\
  Keynote}%
\person{steering/alexander-tormasov}{%
  \href{https://scholar.google.com/citations?user=bsy2_u0AAAAJ}{Alex Tormasov} \\
  \href{https://innopolis.university/en/team-rector/}{Innopolis Uni} \\
  Steering Chair}%
\person{pc/giancarlo-succi}{%
  \href{https://scholar.google.com/citations?user=PdMO57sAAAAJ&hl=en}{Giancarlo Succi} \newline
  \href{https://innopolis.university/en/}{Innopolis Uni} \newline
  PC Chair}%
\person{orgs/yegor-bugayenko}{%
  \href{https://www.yegor256.com/about-me.html}{Yegor Bugayenko} \newline
  \href{https://career.huawei.ru/rri/}{Huawei RRI} \newline
  Orgs Chair}
\vspace{12pt}

We believe that the quality of the source code that millions of programmers
write every day could be much higher than it is now. We believe that the
contribution computer science can make to improve this situation is greatly
undervalued. We aim to solve this problem by gathering
together cutting-edge researchers and letting them share their most recent ideas.

Sponsored by and held at \href{https://innopolis.university/en/}{Innopolis University}.

Topics: Program Analysis, Bug Detection, Maintainability.

Papers will be published in the \textit{Proceedings of ICCQ},
will appear in \href{https://ieeexplore.ieee.org/Xplore/home.jsp}{IEEE Xplore\textsuperscript{\textregistered}},
and will be indexed by Scopus, Web of Science, Google Scholar, DBLP, and others.

\begin{adjustwidth}{1.5in}{0pt}
\vspace{6pt}
Paper/abstract submission date: 31 Dec 2021 \\
Author notification: 1 Mar 2022 \\
Camera-ready submissions: 25 Mar 2022

\vspace{6pt}
ICCQ is also sponsored by:
\href{https://www.hse.ru/en/}{Higher School of Economics},
\href{https://bmstu.ru/en/}{Bauman MSTU},
\href{https://www.huawei.com/}{Huawei},
\href{https://yandex.com/company/}{Yandex},
\href{https://www.linkedin.com/company/btechrussia}{BNP Paribas},
\href{https://www.kaspersky.com/}{Kaspersky},
and \href{https://www.iccq.ru/2022.html#partners}{others}.
\end{adjustwidth}

\newpage

{\color{xred}\bfseries{\Large Student Research Competition (SRC)}}

The competition provides visibility and exposes up-and-coming
researchers to the computer science community. On top of that, the competition g
ives students an opportunity to receive a monetary support from a leading tech
company, motivating them to continue the study.
This year the competition is sponsored by \href{https://career.huawei.ru/rri/}{Huawei RRI}.

You must be a BSc, MSc, or PhD student in any university in Russia:

\begin{enumerate}
\item You pick a research question and work on it;
\item You publish your results in ACM or IEEE conference;
\item You \href{mailto:src@iccq.ru}{email us} a link to your published paper;
\item We let our judges appraise it;
\item We announce winners on the day of the conference;
\item Best papers get a reward (our fund is \$30,000 this year).
\end{enumerate}

More details about the competition, the list of possible research questions,
and the names of our judges are published on \href{https://www.iccq.ru}{iccq.ru}.

\PrintCity{10}{../students}

\end{document}
